\documentclass{article}
\usepackage{hyperref}
\usepackage{fullpage}
\usepackage{xcolor}
\usepackage{xspace}
\xspaceaddexceptions{]\}}% from goo.gl/5RguGu
\newcommand{\toolName}{Pamir~}
\newcommand{\VERSION}{0.1~}
\newcommand{\gitUrl}{https://bitbucket.org/compbio/pamir}
\newcommand{\gitClone}{git@bitbucket.org:compbio/pamir.git}
\newcommand{\comment}[1]{}
\usepackage[utf8]{inputenc}

\usepackage{amssymb}

\title{\toolName Manual}
\author{Pinar Kavak}
\date{\today}

\begin{document}

\maketitle

\tableofcontents
\newpage

\section{Getting Started}

\subsection{Installation}
\toolName can be obtained from \url{\gitUrl}

\subsubsection{Prerequisite}
\toolName relies on specific version of the following tools:
\begin{itemize}
\item g++ 4.9.0 or higher \\
(https://gcc.gnu.org/releases.html)
\item Python 2.7 or higher (needed for the package \texttt{argparse} )
\item boost library 1.57 or higher \\ (https://sourceforge.net/projects/boost/?source=directory)
\item You also need to define the boost library by typing on your shell \\
export BOOST\_INCLUDE= the/BOOST/version/include/ (directory of BOOST in your machine).
\item velvet 1.2.10 or higher
\item BLAST 2.3.0+ or higher
\item Latest BLAST nt database is also needed to be downloaded in dir/to/blast/db (needed for contamination detection). 
\begin{itemize}
\item mkdir dir/to/blast/db
\item cd dir/to/blast/db
\item ../bin/update\_blastdb.pl nt
\end{itemize}
Then
\begin{itemize}
\item ln -s /dir/to/velvet/velveth /dir/to/pamir/velveth
\item ln -s /dir/to/velvet/velvetg /dir/to/pamir/velvetg
\item ln -s /dir/to/blast\_folder /dir/to/pamir/blast
\end{itemize}
\end{itemize}

\subsubsection{Details and Troubleshooting}
\begin{flushleft}
\texttt{git clone --recursive \gitClone}\\
\texttt{cd pamir \&\& make}
\end{flushleft}

\subsection{Running \toolName}
\toolName (Insertion Discovery tool for Whole Genome Sequencing Data) 
detects novel sequence insertions based on one-end anchors (OEA) and orphans from paired-end Whole Genome Sequencing (WGS) reads.

Note that reference genome is required for running \toolName in addition to sequencing or mapping data.

\subsubsection{Project Name}
To run \toolName you have to specify a project name such that \toolName will create a folder to store the results and intermediate files. You need to specify project name by \texttt{-p}. 


\subsubsection{Data Preparation}
\paragraph{Required Information.} Two information are required for running \toolName:
\begin{enumerate}
\item Reference Genome: You need to provide the reference genome in single fasta file by specifying the parameters \texttt{-r} or \texttt{--reference}.

\item Masking File: You can provide a file for masking reference genome. For example,  you can ask \toolName to ignore events in repeat regions by giving \texttt{-m repeat.mask }. When you only want to consider events in genic regions, use  \texttt{-m genic.region -{}-invert-masker} and \toolName will mask those regions not in the given file.

\end{enumerate}

\paragraph{Read Length.} Now \toolName only accepts WGS datasets in which two mates of all reads are of equal length.

\subsubsection{Sequencing Data}
\toolName can take either FASTQ and SAM files as its input. It has three different options to accept inputs:
\begin{itemize}
\item \textbf{SAM/by mrsFAST-best-search}: A paired-end mapping result of your WGS data which satisfies the following conditions: 
    \begin{itemize}
    \item Two mates from a read are grouped together.
    \item All mates are of equal length.
    \end{itemize}
For example, a \textit{best-mapping} SAM be a valid input file for \toolName. 
You can specify by \texttt{--files mrsfast-best-search=wgs.sam}.
You can give multiple best-mapping files too, by comma separated or just the folder directory that includes the inputs. You can specify by\\
\texttt{--files mrsfast-best-search=sample1.sam,sample2.sam,sample3.sam} or \\
\texttt{--files mrsfast-best-search=directory/to/sample\_best\_mapping\_sam\_files/} 

\item \textbf{FASTQ}: \toolName also accepts FASTQ format as the input data once it is a single gzipped file such that two (equal-length) mates of a read locate consecutively. You can specify by giving \texttt{--files fastq=wgs.fastq.gz}.

\item \textbf{Alignment file SAM/BAM}: \toolName also accepts any other alignment output sorted by readname. Alignment output can be in SAM or BAM format. You can specify by \texttt{--files alignment=wgs.sam or --files alignment=wgs.bam}.
\end{itemize}


\subsubsection{MrsFAST Parameters}
\toolName uses mrsFAST for multi-mapping the orphan and OEA reads obtained from the best-mapping output. You can give your own mrsFAST parameters or \toolName will use the default values. Some of the parameters you may want to update are :\\
\begin{itemize}
\item\textbf{--mrsfast-n}: Maximum number of mapping loci of anchor of an OEA. Anchor with higher mapping location will be ignored. 0 for considering all mapping locations. 
(Default = 50)
\item\textbf{--mrsfast-threads}: Number of the threads used by mrsFAST-Ultra for mapping. (Default = 1)
\item\textbf{--mrsfast-errors}: Number of the errors allowed by mrsFAST-Ultra for mapping. In default mode \toolName does not give any error number to mrsFAST-Ultra, in which case it calculates the error value as 0.06 x readlength. (Default = -1)
\item\textbf{--mrsfast-index-ws}: Window size used by mrsFAST-Ultra for indexing the reference genome. (Default = 12)
\end{itemize}

\subsubsection{Other Parameters User can Define}
\begin{itemize}
\item\textbf{--num-worker}: Number of independent prediction jobs to be created. You can define this parameter according to your core number. (Default = 1)
\item\textbf{--resume}: Restart pipeline of an existing project from the stage that has not been completed yet.
\item\textbf{--assembler}: The assembler to be used in orphan assembly stage (Options: velvet, minia, sga. Default = velvet).
\end{itemize}


\subsection{Results}
\toolName generates a VCF file for detected novel sequence insertions. You can run genotyping for each sample after obtaining the VCF file by:


python \textbf{genotyping.py} projectFolder/insertions.out\_wodups\_filtered\_setcov\_PASS.sorted reference.fa.masked sample1\_FASTQ\_1.fq sample1\_FASTQ\_2.fq readlength SAMPLENAME mrsfast-min mrsfast-max projectFolderDirectory TEMP_LEN (1000)

\subsection{Example Commands}
\begin{itemize}

\item To start a new analysis from a mrsfast-best mapping result SAM file:
\begin{flushleft}
\$ ./pamir.py -p my\_project -r ref.fa -{}-files mrsfast-best-search=sample.sam
\end{flushleft}

\item To make a pooled-run with multiple samples separated by comma: 
\begin{flushleft}
\$ ./pamir.py -p my\_project -r ref.fa  -{}-files mrsfast-best-search=sample.sam,sample2.sam,sample3.sam
\end{flushleft}

\item To make a pooled-run with multiple samples which are in a folder called SAMPLEFOLDER: 
\begin{flushleft}
\$ ./pamir.py -p my\_project -r ref.fa  -{}-files mrsfast-best-search=SAMPLEFOLDER
\end{flushleft}

\item To start from another mapping tool's alignment result SAM/BAM file:
\begin{flushleft}
\$ ./pamir.py -p my\_project -r ref.fa -{}-files alignment=sample.bam
\end{flushleft}

\item To start from a gzipped fastq file,
\begin{flushleft}
\$ ./pamir.py -p my\_project -r ref.fa -{}-files fastq=sample.fastq.gz
\end{flushleft}

\item To ignore regions in a mask file (e.g., repeat regions), 
\begin{flushleft}
\$ ./pamir.py -p my\_project -r ref.fa -m repeat.txt -{}-files mrsfast-best-search=sample.sam
\end{flushleft}

\item To analyze events only in some regions of the reference genome (e.g., genic regions), 
\begin{flushleft}
\$ ./pamir.py -p my\_project -r ref.fa -m genic.region -{}-invert-mask  -{}-files mrsfast-best-search=sample.sam
\end{flushleft}

\item To make sure that mrsFAST will not report the mapping locations of an OEA read more after the 30th location:
\begin{flushleft}
\$ ./mistrvar.py -p my\_project -r ref.fa -{}-mrsfast-n 30 -{}-files mrsfast-best-search=sample.sam
\end{flushleft}

\item To specify the core number for mrsFAST during multi-mapping of OEAs: 
\begin{flushleft}
\$ ./pamir.py -p my\_project -r ref.fa  -{}-mrsfast-threads 8 -{}-files mrsfast-best-search=sample.sam
\end{flushleft}

\item To speed up the prediction process by defining the independent prediction jobs according to available core numbers: 
\begin{flushleft}
\$ ./pamir.py -p my\_project -r ref.fa  -{}-num-worker 20 -{}-files mrsfast-best-search=sample.sam
\end{flushleft}

\item To specify the assembler as sga for orphan assembly and also the number of prediction jobs will be 20: 
\begin{flushleft}
\$ ./pamir.py -p my\_project -r ref.fa  -{}-num-worker 20 -{}-assembler sga -{}-files mrsfast-best-search=sample.sam
\end{flushleft}

\item To resume from the previously finished stage: 
\begin{flushleft}
\$ ./pamir.py -p my\_project -{}-resume
\end{flushleft}

\end{itemize}

\subsection{Example Invalid Commands}
The following commands do not satisfy requirements of \toolName and will fail pamir.py:

\begin{itemize}
\item Project name is missing:
\begin{flushleft}
\$./pamir.py -r ref.fa -{}-files alignment=sample.sam
\end{flushleft}

\item  Reference genome file is missing:
\begin{flushleft}
\$./pamir.py -p my\_project -{}-files alignment=sample.sam
\end{flushleft}

\item Incorrect path of the mask file:
\begin{flushleft}
\$./pamir.py -p my\_project -m non-exist-mask-file -{}-files alignment=sample.sam
\end{flushleft}

\item No input sequencing files:
\begin{flushleft}
\$./pamir.py -p my\_project -r ref.fa
\end{flushleft}

\item Multiple sequencing sources:
\begin{flushleft}
\$./pamir.py -p my\_project  -r ref.fa -{}-files mrsfast-best-search=sample.sam fastq=sample2.fastq.gz
\end{flushleft}

\end{itemize}



%%%%%%%%%%%%%%%%%%%%%%%%
% Comment the following part before completion
%%%%%%%%%%%%%%%%%%%%%%%%
% More Advanced Materials on Usage
% \section{Pipeline and Intermediate Files}
% \subsection{Filtering Predictions}
% \texttt{allinone\_filtering.py}
% \subsection{Set Cover}


% \section{Misc}

\end{document}
